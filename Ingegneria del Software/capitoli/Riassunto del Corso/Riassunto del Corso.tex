\myparagraph{
Il corso di Ingegneria del Software approfondisce tematiche viste nel corso di Analisi e Progettazione del Software,
ovviamente introducendo nuovi concetti mai analizzati nel corso del secondo anno.

Il corso è molto interessante, soprattutto perchè, rispetto ad APS, viene trattato molto il lato pratico, quindi non ci
si sofferma a definire le classi in Java ma anche i loro comportamenti e relazioni con le altre classi. Il mio consiglio è
quello di non vedere il corso come un APS 2.0, solo la parte introduttiva rispolvera le vecchie nozioni, per il resto sono
argomenti mai trattati sino ad'ora.

I professori del corso sono:
\begin{enumerate}
    \item Per la teoria: \textbf{Francesca Arcelli Fontana}
    \item Per le esercitazioni: \textbf{Oliviero Riganelli}
\end{enumerate}

In questo file che metto a disposizione riassumerò tutta la teoria del corso + le esercitazioni, consiglio sempre di accompagnare
il file con le slide che mettono a disposizione i prof, (\textbf{soprattutto la parte di esercitazione}), dove ho scritto la teoria,
ma non ho inserito alcun codice se non i diagrammi fondamentali, può accadere che mi sia perso qualcosa per strada sicuramente\dots

Riguardo la modalità di esame, è simile al preappello di APS, quindi un progetto (ovviamente più esteso) + l'orale, i prof vi
daranno indicazioni a riguardo.
\vspace{3cm}

\begin{center}
Detto questo, in bocca al lupo se sceglierete questo corso e auguro a tutti 

\textit{una vita meravigliosa}
\end{center}
}

\newpage