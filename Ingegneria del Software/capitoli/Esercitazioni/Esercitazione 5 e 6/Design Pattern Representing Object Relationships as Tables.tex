\mysubsectionformatted{Design Pattern Representing Object Relationships as Table}
\myparagraph{
    \begin{tcolorbox}[colback=blue!5!white, colframe=blue!75!black]
        Il pattern, come dice il nome, ci permette di rappresentare le associazioni tra le tabelle in un
        database relazionale.
        Le associazioni sono:
        \begin{enumerate}
            \item \textbf{Uno ad uno (1:1)}: tra due tabelle, inseriamo una chiave esterna in una o entrambe le tabelle in modo
            da rappresentare gli oggetti in relazione. Si può anche creare una terza tabella che registra le chiavi esterne per
            ciascun oggetto in relazione.
            \item \textbf{Uno a molti (1:n)}: crea una tabella che registra le chiavi esterne degli oggetti in relazione.
            \item \textbf{Molti a molti (n:n)}: crea una tabella che registra le chiavi esterne degli oggetti in relazione.
        \end{enumerate}
    \end{tcolorbox}
}