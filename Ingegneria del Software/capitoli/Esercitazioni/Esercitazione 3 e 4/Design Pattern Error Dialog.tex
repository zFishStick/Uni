\mysubsectionformatted{Design Pattern Error Dialog}
\myparagraph{
    \begin{tcolorbox}[colback=blue!5!white, colframe=blue!75!black]
        Il pattern permette di utilizzare un oggetto non appartenente all'interfaccia utente
        e accessibile attraverso un Singleton con lo scopo di comunicare gli errori agli utenti.
        Il pattern "avvolge" uno o più oggetti della UI (es. una finestra di testo), delegandogli
        la notifica dell'errore. Inoltre, ripoterà l'errore al logger centralizzato degli errori.
        Una \textit{Factory} ha il compito di creare l'oggetto UI appropriato in base ai parametri
        di sistema ricevuti, quindi in base all'errore.
    \end{tcolorbox}
}

\newpage