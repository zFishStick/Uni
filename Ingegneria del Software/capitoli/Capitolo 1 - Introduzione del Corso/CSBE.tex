\mysubsectionformatted{CBSE - Component Based Software Engineering}
\myparagraph{
    Le parole chiave che determinano questo approccio sono la riusabilità e l'uso di componenti software.
    \\
    Questo approccio nasce dal momento che molto spesso lo sviluppo orientato agli oggetti fallisca per mancanza di riuso
    dei componenti, ciò è causato da vari motivi:

    \begin{enumerate}
        \item Le classi sono troppo dettagliate e specifiche per poter essere riutilizzate in altri contesti.
        \item I componenti sono più astratti delle classi stesse.
        \item I componenti vengono visti come dei provider di servizi indipendenti (stand alone).
        \item I componenti possono esistere come delle entità indipendenti.
    \end{enumerate}

    \newpage

    \noindent L'approccio CBSE fa sì che i componenti forniscano una funzionalità senza dover considerare dove il componente viene
    eseguito o il suo linguaggio di programmazione, questo perché è un'entità eseguibile e indipendente che può essere usato per
    uno o più oggetti eseguibili. L'interfaccia del componente è pubblica e tutte le interazioni avvengono tramite questa.
    \\
    Come per i diagrammi, questo approccio dispone di due tipi di interfacce:
    \vspace{-12pt}

    \begin{center}
        \resizebox{\textwidth}{!}{
            \begin{tabular}{|l|l|}
                \hline
                \textbf{Interfacce Fornite}   &
                \begin{tabular}[c]{@{}l@{}}Definiscono i servizi forniti dal componente verso altri componenti.\\ Essenzialmente, è l'API che definisce i metodi che possono essere chiamati dall'utente.\end{tabular}
                \\ \hline
                \textbf{Interfacce Richieste} &
                \begin{tabular}[c]{@{}l@{}}Definiscono i servizi richiesti dal componente per eseguire quanto specificato.\\ Ciò non compromette l'indipendenza o la distribuzione del componente perché l'interfaccia\\ richiesta non definisce come questi servizi vengono forniti.\end{tabular}
                \\ \hline
            \end{tabular}
        }
    \end{center}

    \vspace{8pt}

    \mysubsubsectionformatted{Componenti essenziali del CBSE}
    \begin{enumerate}
        \item Componenti indipendenti specificati dalle loro interfacce.
        \item Standard specifici dei componenti per facilitare la loro integrazione.\\
              Stabiliscono come i componenti comunicano e operano tra loro.
        \item Middleware che fornisce supporto per la portabilità dl componente.
        \item Un processo di sviluppo mirato al riuso.
    \end{enumerate}

    \mysubsubsectionformatted{Principi di progettazione}
    \begin{enumerate}
        \item I componenti sono indipendenti, non inteferiscono tra loro.
        \item L'implementazione dei componenti è nascosta.
        \item La comunicazione avviene tra interfacce ben definite.
        \item Le piattaforme dei componenti sono condivise in modo da ridurre i costi di sviluppo.
    \end{enumerate}

    I componenti sviluppati tramite diversi approcci NON lavorano tra loro.

    \mysubsubsectionformatted{Elementi basici di un modello di componente}
    \begin{center}
        \resizebox{\textwidth}{!}{%
            \begin{tabular}{|l|l|}
                \hline
                \textbf{Interfacce}    & \begin{tabular}[c]{@{}l@{}}Il modello dei componenti specifica come le interfacce devono essere definite\\ e i suoi elementi (operazioni, nomi, parametri...).\end{tabular}                                    \\ \hline
                \textbf{Uso}           & \begin{tabular}[c]{@{}l@{}}Per essere distribuiti e accessi in remoto, ciascun componente deve avere un\\ nome univoco.\end{tabular}                                                                           \\ \hline
                \textbf{Distribuzione} & \begin{tabular}[c]{@{}l@{}}Il modello dei componenti include una specifica di come i componenti \\ devono essere impacchettati per la loro distribuzione come entità\\ indipendenti e eseguibili.\end{tabular} \\ \hline
            \end{tabular}%
        }
    \end{center}

    \newpage
    \mysubsubsectionformatted{Processi del CBSE}
    I processi del CBSE sono, come dice il nome, dei processi software che supportano l'utilizzo di
    tale approccio. Permettono la riusabilità delle attività di processo coinvolte nelle attività
    di sviluppo e un riuso dei componenti.
    \\
    Bisogna fare una distinzione in merito alla modalità di sviluppo e il riuso:
    \vspace{-12pt}
    \begin{center}
        \resizebox{\textwidth}{!}{
            \begin{tabular}{|l|l|}
                \hline
                \textbf{Sviluppo PER il riuso} & \begin{tabular}[c]{@{}l@{}}Si basa sullo sviluppo di componenti che \\ verranno poi utilizzati in altre applicazioni.\end{tabular} \\ \hline
                \textbf{Sviluppo CON riuso}    & \begin{tabular}[c]{@{}l@{}}Si basa sullo sviluppo di nuove applicazioni\\ usando dei componenti già esistenti.\end{tabular}        \\ \hline
            \end{tabular}%
        }
    \end{center}
    \vspace{8pt}

    \mysubsubsectionformatted{Processi di supporto del CBSE}
    \begin{center}
        \resizebox{\textwidth}{!}{%
            \begin{tabular}{|l|l|}
                \hline
                \textbf{Acquisizione del componente}   & \begin{tabular}[c]{@{}l@{}}Processo che acquisisce i componenti per il loro riuso o trasforma\\ lo sviluppo in un componente riutilizzabile.\end{tabular}                        \\ \hline
                \textbf{Gestione del componente}       & \begin{tabular}[c]{@{}l@{}}Verte sulla gestione dei componenti riusabili assicurandosi\\ siano propriamente catalogati, immagazzinati e resi disponibili per il riuso.\end{tabular} \\ \hline
                \textbf{Certificazione del componente} & Processo che controlla se un componente soddisfa le sue specifiche.                                                                                                                 \\ \hline
            \end{tabular}%
        }
    \end{center}
    \vspace{8pt}

    \mysubsubsectionformatted{CBSE per il riuso}
    Si concentra sullo sviluppo dei componenti. Lo scopo è quello di generalizzare quei componenti che vengono creati per 
    specifiche applicazioni con lo scopo di renderli riutilizzabili.
    \\
    Un componente ha più probabilità di essere riutilizzato se è associato a un'astrazione di dominio stabile, ovvero a un oggetto
    di business ben definito e costante nel sistema.
    \\
    Per fare un esempio di dominio stabile può essere un ospedale, dove i componenti hanno ciascuno degli scopi fontamentali 
    (dottori, pazienti, trattamenti ecc\dots)

    \newpage
    \mysubsubsectionformatted{CBSE con riuso}
    Al contrario della prima tipologia, questa cerca e integra componenti già esistenti.
    Quando si riutilizzano i componenti in un progetto software, è fondamentale considerare i compromessi tra i requisiti ideali 
    di ciò che si desidererebbe avere e i servizi reali forniti dai componenti disponibili.
    \\
    Questo approccio richiede:
    \begin{enumerate}
        \item Sviluppo dei requisiti preliminari (essenziali).
        \item Ricerca dei componenti e la loro modifica in base alle funzionalità disponibili.
        \item Cercare nuovamente per trovare componenti migliori che soddifsano i requisiti.
        \item Unire (o comporre) i componenti per creare il sistema.
    \end{enumerate}

    \mysubsubsectionformatted{Problemi ancora aperti con CBSE}
    \begin{enumerate}
        \item Quanto è affidabile un componente senza un codice sorgente disponibile?
        \item Chi certifica la qualità dei componenti?
        \item Come possono essere predette le proprietà emergenti della composizione dei componenti?
        \item Come si fanno le analisi tra le caratteristiche di un componente con un altro?
    \end{enumerate}
}
    \newpage