\mysubsectionformatted{Pattern of Enterprise Applications Architectures}
\myparagraph{
    All'interno di grandi aziende, si hanno una grandissima quantità di dati da gestire,
    organizzati in sistemi moderni come i DBMS, e nel tempo questi dati potranno dover essere
    migrati attraverso nuove applicazioni, si parla di dati persistenti su archi di tempo molto ampi.
    \\
    L'accesso a questi dati è concorrente, arrivando fino a milioni di accessi contemporanei, le interfacce
    sono molto articolate, distinte tra loro e presentate in modo differente per tipi di utenti e obiettivi
    differenti.
    \\
    Le applicazioni \textbf{enterprise} possono essere integrate con ulteriori applicazioni. Queste fanno uso
    di architetture stratificate (per la maggior parte di essi).
    
    \coloredtext[blue]{\textbf{Business Logic:}} regole di interpretazione e elaborazione dei dati in base
    all'utilità dell'applicazione. Queste regole tendono a cambiare in base a diversi criteri:
    \begin{itemize}
        \item Gestione personalizzata di clienti
        \item Adeguamento a nuove logiche di mercato
        \item Cambiamento nelle strategie di marketing
        \item Altro
    \end{itemize}

    Riguardo la \textbf{performance}, molte decisioni architetturali aggravano su essa, portando a sacrificarne
    un po' pur di migliorare la comprensibilità.\\
    Tra i fattori che costituiscono la performance abbiamo la \textbf{scalabillità},
    \\l'\textbf{efficienza}, il \textbf{Tempo di Risposta} e la \textbf{Latenza} (Quanto puoi fare in un certo lasso di tempo).
    
    \mysubsubsectionformatted{Design Patterns e Architetturali}
    Per fare una distinzione tra i due:
    \begin{tcolorbox}[colback=blue!5!white, colframe=blue!75!black]
        I \textbf{Design Pattern} aiutano a trovare delle soluzioni a problemi in modo tale che si possano usare queste soluzioni
        tantissime volte. Purtroppo adattarli ai casi reali richiede uno sforzo progettuale significativo.
    \end{tcolorbox}

    \begin{tcolorbox}[colback=green!5!white, colframe=green!75!black]
        I \textbf{Pattern Architetturali} offrono soluzioni a problemi architetturali, aiutando a documentare le decisioni prese,
        facilitando la comunicazione tra gli stakeholder mediante un vocabolario comune.
    \end{tcolorbox}
    \newpage
    \mysubsubsectionformatted{Tipologie di Pattern Architetturali}
    \begin{enumerate}
        \item Domain Logic Patterns
        \item Data Source Architectural Patterns
        \item Object Relational Behavioral Patterns
        \item Object Relational Structural Patterns
        \item Web Presentation Patterns
        \item Concurrency Patterns
        \item Security Patterns
              \begin{enumerate}
                \renewcommand{\labelenumii}{\theenumi.\arabic{enumii}}
                \setcounter{enumii}{0}
                  \item Federated Identity Pattern
                  \item Gatekeeper Pattern
                  \item Valet Key Pattern
              \end{enumerate}
        \item Performance Patterns
    \end{enumerate}
    Per descrivere un pattern architetturale ci focalizzeremo su 4 punti:
    \begin{enumerate}
        \item \textbf{Descrizione}
        \item \textbf{Diagramma UML}
        \item \textbf{Esempio di applicazione}
        \item \textbf{Vantaggi e Svantaggi}
    \end{enumerate}
    \mysubsubsectionformatted{Frameworks}
    I framework sono delle librerie di codice astratte e estendibili che vengono adattate per scopi specifici da parte
    degli sviluppatori. Per i software moderni sono oramai essenziali, migliorano la produttività ma con degli effetti
    collaterali indesiderati, come debolezze architetturali o vulnerabilità nella sicurezza.
    \\
    Tra i framework più conosciuti abbiamo
    \begin{itemize}
        \item Maven
        \item Django
        \item Spring
        \item Molti altri
    \end{itemize}

    Tra i produttori di framework troviamo anche \textbf{Amazon Web Services (AWS)} e \textbf{Microsoft's Azure}.
    
    \newpage
}