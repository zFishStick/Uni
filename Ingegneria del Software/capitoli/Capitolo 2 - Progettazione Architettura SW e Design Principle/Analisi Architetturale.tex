\mysubsectionformatted{Analisi Architetturale}
\myparagraph{
    L'analisi architetturale ha l'obiettivo di indentificare e risolvere i requisiti non
    funzionali del sistema (es. la sicurezza) nel contesto dei requisiti funzionali.
    \\
    Ricordiamo il documento che definisce i requisiti non funzionali, ovvero
    le \\ \coloredtext[blue]{specifiche supplementari}.
    \\
    I passi da seguire durante l'analisi sono:
    \begin{enumerate}
        \item \textbf{Investigazione}: identificazione di requisiti funzionali e non funzionali
              che hanno maggiore impatto sul sistema.
        \item \textbf{Progettazione}: risoluzione dei requisiti identificati.
    \end{enumerate}
    Per i requisiti che hanno impatto architetturale significativo si analizzano le
    alternative e si creano delle soluzioni per gestire i compromessi e le priorità.

    \mysubsubsectionformatted{Collezione e organizzazione dei requisiti non funzionali}
    Viene stilata la seguente tabella per i requisiti non funzionali:
    \vspace{-0.1cm}

    \begin{center}
        \renewcommand{\arraystretch}{1.5}
        \resizebox{\columnwidth}{!}{%
            \begin{tabular}{|ll|}
                \hline
                \multicolumn{2}{|c|}{\cellcolor[HTML]{3531FF}{\color[HTML]{FFFFFF} \textbf{Tabella dei fattori}}}                                                                                               \\ \hline
                \multicolumn{1}{|l|}{\textbf{Nome fattore}}                & Nome del fattore (requisito)                                                                                                                \\ \hline
                \multicolumn{1}{|l|}{\textbf{Misure e scenari di qualità}} & Come il fattore sarà misurato e valutato tramite metriche di qualità                                                                        \\ \hline
                \multicolumn{1}{|l|}{\textbf{Variabilità}}                 & \begin{tabular}[c]{@{}l@{}}Quanto il sistema è flessibile a soddisfare il requisito \\ con possibilità di evoluzione in futuro\end{tabular} \\ \hline
                \multicolumn{1}{|l|}{\textbf{Impatto del fattore}}         & Quanto influisce il fattore sull'architettura del sistema                                                                                   \\ \hline
                \multicolumn{1}{|l|}{\textbf{Priorità per il successo}}    & Quanto è importante il fattore per il successo del sistema                                                                                  \\ \hline
                \multicolumn{1}{|l|}{\textbf{Difficoltà o rischi}}         & Possibili difficoltà associate al soddisfacimento di questo requisito                                                                       \\ \hline
            \end{tabular}%
        }
    \end{center}
    

    \mysubsubsectionformatted{Gerarchia di obiettivi per le decisioni architetturali}
    Esistono 3 tipi di obiettivi e priorità:
    \begin{enumerate}
        \item Vincoli inflessibili: sicurezza, conformità alle leggi\dots
        \item Obiettivi di business: tipi di clienti finali interessati al prodotto software
        \item Altri obiettivi: estensioni con altre funzionalità
    \end{enumerate}

    \mysubsubsectionformatted{Principi di base della progettazione architetturale}
    Sono dei principi che offrono maggiore efficienza all'architettura software:
    \begin{center}
        \resizebox{0.4\columnwidth}{!}{%
            \begin{tabular}{|c|}
                \hline
                \rowcolor[HTML]{3166FF}
                \multicolumn{1}{|c|}{\cellcolor[HTML]{3166FF}{\color[HTML]{FFFFFF} \textbf{Principi}}}               \\ \hline
                Accoppiamento basso                                                                                  \\ \hline
                Coesione alta                                                                                        \\ \hline
                Variazione protetta                                                                                  \\ \hline
                \begin{tabular}[c]{@{}l@{}}Separazione degli interessi e \\ localizzazione dell'impatto\end{tabular} \\ \hline
                \begin{tabular}[c]{@{}l@{}}Utilizzo di pattern e \\ stili architetturali\end{tabular}                \\ \hline
            \end{tabular}%
        }
    \end{center}

    \mysubsubsectionformatted{L'architettura logica}
    Questo tipo di architettura vede il sistema come un insieme di package logici. Descrive il sistema nei termini della
    sua organizzazione in layers, packages, frameworks, classi, interfacce e sottosistemi.
    \\
    I\textbf{ \coloredtext[blue]{package}} raggruppano un insieme di responsabilità coese
    (strettamente correlate tra loro), questa caratteristica viene definita anche come
    \textbf{ \coloredtext[blue]{modularizzazione}}.
    \\
    Questa caratteristica favorisce una separazione degli interessi (separation of concerns).
    \\
    Ciascun package si occupa di quello per cui è stato progettato, senza preoccuparsi del resto del sistema.

    \mysubsubsectionformatted{I pattern}
    I pattern si dividono in diverse tipologie:
    \vspace{-0.5cm}
    \begin{center}
        \resizebox{\columnwidth}{!}{%
            \begin{tabular}{|
                    >{\columncolor[HTML]{3166FF}}l |l|}
                \hline
                {\color[HTML]{FFFFFF} \textbf{Pattern architetturali}} & Progettazione a larga scala (es. i pattern layers)                                                                                                                                    \\ \hline
                {\color[HTML]{FFFFFF} \textbf{Design patterns}}        & \begin{tabular}[c]{@{}l@{}}Progettazione a scala media-piccola che hanno\\ connessioni fra gli elementi a larga scala (es. i pattern GoF)\end{tabular}                                \\ \hline
                {\color[HTML]{FFFFFF} \textbf{Idiomi}}                 & \begin{tabular}[c]{@{}l@{}}Soluzioni progettuali di basso livello legate al linguaggio\\ o alle implementazioni usate (es. comparazione di due stringhe,\\ Singleton...)\end{tabular} \\ \hline
                {\color[HTML]{FFFFFF} \textbf{Strategie}}              & \begin{tabular}[c]{@{}l@{}}Direttive e consigli pratici su come risolvere un certo problema\\ in modo efficace.\end{tabular}                                                          \\ \hline
            \end{tabular}%
        }
    \end{center}
    Per chiarire meglio la differenza tra i pattern architetturali e i design pattern ci si può riferire alle
    seguenti domande:
    \begin{itemize}
        \item Quali sono le parti fondamentali del sistema? \rightarrow \vspace{0.25em} Pattern architetturali
        \item Come sono connesse tra loro? \rightarrow \vspace{0.25em} Design patterns
    \end{itemize}

    \mysubsubsectionformatted{Problemi nelle architetture software}
    Si possono verificare diverse problematiche durante lo sviluppo dell'architettura software:
    \begin{enumerate}
        \item Alto accoppiamento dei componenti, portano alla modifica di più elementi in caso di cambiamento.
        \item La logica applicativa condivisa con le GUI.
        \item I servizi tecnici direttamente collegati alla logica applicativa, sfavorendone il riuso.
        \item L'evoluzione del sistema risulta difficoltosa dato l'alto accoppiamento e le numerose funzionalità
              distinte e non correlate tra loro.
        \item Ovviamente possono esserci tante altre problematiche\dots
    \end{enumerate}
}